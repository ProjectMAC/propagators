\documentclass[12pt,letterpaper]{article}

%%%% Alexey's general-purpose LaTeX stuff

%% Paper size is a persistent annoyance.  The (current) suggested
%% strategy is to always use letterpaper or a4paper options to
%% documentclass, and see to it that \pdfpagewidth and \pdfpageheight
%% get set correctly.  The color package, for example, does the
%% latter.

%% Also be sure to view the document at the right paper size with
%% xdvi, and print it at the right paper size with dvips.

\usepackage{amsmath}
\usepackage{amsthm}
\usepackage{amsfonts}
\usepackage{amssymb}

\usepackage{epsfig}
\usepackage{graphicx}
\usepackage[usenames]{color}

\usepackage{ifpdf}

\newtheorem{theorem}{Theorem}[section]
\newtheorem{lemma}[theorem]{Lemma}
\newtheorem{corollary}[theorem]{Corollary}
\newtheorem{statement}[theorem]{Statement}
\newtheorem{conjecture}[theorem]{Conjecture}

\ifpdf
  \DeclareGraphicsRule{.pdftex}{pdf}{.pdftex}{}
\else
\fi

%%% Commands
\newcommand{\code}[1]{{\tt #1}}
\newcommand{\defn}[1]{{\bfseries{#1}}}
\newcommand{\todo}[1]{{\bf *** TODO: {#1} ***}}
%% The idea here is to mark todo items that will not be showstoppers
%% if the todo comment just vanishes.
\newcommand{\otodo}[1]{{\bf *** TODO (Optional): {#1} ***}}
% \renewcommand{\otodo}[1]{{}}  % Like so.

\def\caret{\^\,\hspace{-2pt}}

%%% Shortcuts
\newcommand{\be}{\begin{equation}}  %numbered equations
\newcommand{\ee}{\end{equation}}
\newcommand{\bea}{\begin{eqnarray}}
\newcommand{\eea}{\end{eqnarray}}
\newcommand{\beas}{\begin{eqnarray*}}
\newcommand{\eeas}{\end{eqnarray*}}
\newcommand{\bl}{\begin{lisp}}
\newcommand{\el}{\end{lisp}}
\newcommand{\bls}{\begin{lisp*}}
\newcommand{\els}{\end{lisp*}}
\newcommand{\bv}{\begin{verbatim}{}}
\newcommand{\ev}{\end{verbatim}{}}
\newcommand{\LRa}{\ensuremath{\Leftrightarrow}}
\newcommand{\Ra}{\ensuremath{\Rightarrow}}
\newcommand{\La}{\ensuremath{\Leftarrow}}

%%% Pictures
\newlength{\pagepic}
\setlength{\pagepic}{390pt}

\newlength{\halfpagepic}
\setlength{\halfpagepic}{210pt}

\newcommand{\magicpicture}[1]{%
\ifpdf
  \input{pictures/#1.pdftex_t}
\else
  \input{pictures/#1.pstex_t}
\fi}

\newcommand{\diagram}[2]{%
\begin{figure*}
\begin{center}
\includegraphics[width=\pagepic]{pictures/#1}
\caption{{{#2}}}
\label{fig:#1}
\end{center}
\end{figure*}
}

\newcommand{\magicdiagram}[2]{%
\begin{figure*}
\begin{center}
\magicpicture{#1}
\caption{{{#2}}}
\label{fig:#1}
\end{center}
\end{figure*}
}

\newcommand{\twocaptiondiagram}[3]{%
\begin{figure*}
\begin{center}
\magicpicture{#1}
\caption[{{#2}}]{{{#3}}}
\label{fig:#1}
\end{center}
\end{figure*}
}

%%% Floating controls.  See discussion at
%%% http://dcwww.camd.dtu.dk/~schiotz/comp/LatexTips/LatexTips.html
\renewcommand{\topfraction}{0.85}
\renewcommand{\textfraction}{0.1}
\renewcommand{\floatpagefraction}{0.75}

\author{Alexey Radul}
\title{Composing Partial Information}

\begin{document}

\maketitle

The propagator structure is parametrized by the type of partial
information being propagated.

This parametrization determines how cells merge the partial
information, and how operations that are not inherently aware of the
partial information react to it.  For example, addition is not
inherently aware of the partialness of its arguments, nor does it have
any specific interactions with any partial information structures
(except, sort of, interval arithmetic).  Therefore, addition should
react to, for example, truth maintenance systems, in a mechanical way,
that depends on what the nature of truth maintenance is, but not on
the nature of addition.

There is one other place in the body of knowledge that I am aware of
where some system is parametrized over some structure, such that most
operations are translated uniformly in accordance with the
parametrizing structure, and independent of the particular operation.
That place is interpreters parametrized by monads.\cite{?}  Indeed,
the parametrization of propagators over partial information structures
is very similar to the parametrization of interpreters over monads;
and the \code{generic-unpack} and \code{generic-flatten} procedures
are an attempt to mimic monads in the propagator setting.

There are some significant differences between monads and partial
information structures.  First, the latter are required to support a
merge operation for updating the contents of cells; and the
computations performed by propagators operating within a particular
partial information structure are ``required'' to be monotonic with
respect to the lattice induced by that merge operation.  \todo{Get
  into the partial information structure laws, by analogy with the
  monad laws?  How were the monad laws derived?  What theorems follow
  from the partial information structure laws?}

Second, partial information structures as currently embraced by the
system need not be fully parametric over their ``contents''.  For
example, merging intervals relies on being able to compare the
endpoints of intervals with one another, so intervals are not fully
polymorphic over the type of their endpoints.  (Also, interval
arithmetic is not entirely uniform over the operations being
performed, in that multiplying by a negative number is
inverse-monotonic in the underlying order on the endpoints).

Nonetheless, monads and partial information structures appear to have
an important, if unfortunate, property in common: they do not compose.
Monad transformers \cite{?} are a story for augmenting monads with
additional structure so as to be able to compose them.  The approach
is promising, but confusing and difficult to grasp and follow.
(Monads themselves were bad enough!)  The approach would furthermore
require considerable effort to adapt to partial information
structures, because the latter are, as discussed earlier, just
analogous to monads, not the same.  Therefore, having chased the
question of monad composition, and partial information composition,
for long enough, I choose to punt it.

What is one to do in a world of non-composing partial information
structures?  The same thing one does in a world of non-composing
monads---pick one and stick with it.  In their early
conceptions,\cite{wadler?, moggi?}  monads were thought of as parametrizing
entire programming language interpreters, and therefore the particular
monad chosen would form the global set of features available in a
particular interpreted language \todo{Is this assertion actually
  true?}.  This was still an advance over the non-monadic approach,
because at least the basic pure interpreter, and perhaps some ad-hoc
pieces of the implementations of the various monads, could be shared.
In time, the \code{do} notation was invented \cite{??}, and made it
possible to delineate a part of a program as being in a monad, so that
different, sufficiently non-interacting parts of the same program
could be in different monads (or different instances of the same
monad).  This was the great advance that made the monadic style
practicable in a full programming language (e.g. Haskell).

The appropriate equivalent of \code{do} notation for propagators has
not been invented yet, nor has the underlying mechanism for
segregating one partial information structure from another so that
different ones can be used in different parts of a system without
interference.  I expect such an invention will be necessary before the
propagator language becomes really practical, but until then you'll
just have to make do with what I've got.

What have I got?  Think of the full collection of methods defined on
all the generic-foo operations at any one time as defining one big
partial information structure.  For example, if you load all of
\code{core}, you will get a propagator machine whose partial information
structure can handle
\begin{itemize}
\item The \code{nothing} object, meaning ``no information''
\item Raw Scheme values, meaning exactly themselves
\item Intervals over positive numbers, meaning ``it's within these bounds''
\item Supported values, meaning ``If you believe the given premises, you have information X''
\item Truth maintenance systems, which store that same information but
  admit (even automatic) changes in the set of believed premises.
\item Compositions: TMSes over intervals, TMSes over raw values,
  supported intervals, and supported raw values.
\end{itemize}
If you also load \code{extensions/symbolics}, this will be augmented
with symbolic expressions over real numbers, with tracking and
solution of symbolic equations between them; and all sensible
compositions thereof (though maybe not compositions of symbolic
expressions and intervals).  If you also load
\code{extensions/symbolics-ineq}, this will be further augmented with
symbolic boolean values, and tracking and solutions of inequalities.

On the other hand, the virtual-copies data structures in
\code{extensions/environments}, for example, do not interoperate at
all well with the TMSes from \code{core}.  So if you load that, you
effectively have two incompatible partial information structures: one
where TMSes are allowed and virtual copies are not, and one where
virtual copies are allowed and TMSes are not.  This is not inherently
a problem, but there is no static notation like the \code{do}-notation
for separating the regions of propagator network where the two
different structures are acceptable, so you must manually arrange for
the two kinds of things not to meet in any unpleasant places.

Unfortunately, the definitions of the generic operations used in the
propagator system are global.  So while it would in principle be
possible to define the generic operations to handle either the partial
information structure ``virtual copies of truth maintenance systems of
X'', or the partial information structure ``truth maintenance over
virtual copies of X'', there is no way to allow both in the same
propagator network, even if they never intersect.  At least I have not
yet invented any way to allow this.  Though perhaps the monad
transformer story can help arrange it.

As a consequence, you must yourself arrange, by choosing which method
definitions you actually invoke for a particular propagator system,
which partial information structure you are using.  A survey of the
compatibilities of the different ones remains to be done.
\end{document}
