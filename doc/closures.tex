\documentclass[12pt,letterpaper]{article}

%%%% Alexey's general-purpose LaTeX stuff

%% Paper size is a persistent annoyance.  The (current) suggested
%% strategy is to always use letterpaper or a4paper options to
%% documentclass, and see to it that \pdfpagewidth and \pdfpageheight
%% get set correctly.  The color package, for example, does the
%% latter.

%% Also be sure to view the document at the right paper size with
%% xdvi, and print it at the right paper size with dvips.

\usepackage{amsmath}
\usepackage{amsthm}
\usepackage{amsfonts}
\usepackage{amssymb}

\usepackage{epsfig}
\usepackage{graphicx}
\usepackage[usenames]{color}

\usepackage{ifpdf}

\newtheorem{theorem}{Theorem}[section]
\newtheorem{lemma}[theorem]{Lemma}
\newtheorem{corollary}[theorem]{Corollary}
\newtheorem{statement}[theorem]{Statement}
\newtheorem{conjecture}[theorem]{Conjecture}

\ifpdf
  \DeclareGraphicsRule{.pdftex}{pdf}{.pdftex}{}
\else
\fi

%%% Commands
\newcommand{\code}[1]{{\tt #1}}
\newcommand{\defn}[1]{{\bfseries{#1}}}
\newcommand{\todo}[1]{{\bf *** TODO: {#1} ***}}
%% The idea here is to mark todo items that will not be showstoppers
%% if the todo comment just vanishes.
\newcommand{\otodo}[1]{{\bf *** TODO (Optional): {#1} ***}}
% \renewcommand{\otodo}[1]{{}}  % Like so.

\def\caret{\^\,\hspace{-2pt}}

%%% Shortcuts
\newcommand{\be}{\begin{equation}}  %numbered equations
\newcommand{\ee}{\end{equation}}
\newcommand{\bea}{\begin{eqnarray}}
\newcommand{\eea}{\end{eqnarray}}
\newcommand{\beas}{\begin{eqnarray*}}
\newcommand{\eeas}{\end{eqnarray*}}
\newcommand{\bl}{\begin{lisp}}
\newcommand{\el}{\end{lisp}}
\newcommand{\bls}{\begin{lisp*}}
\newcommand{\els}{\end{lisp*}}
\newcommand{\bv}{\begin{verbatim}{}}
\newcommand{\ev}{\end{verbatim}{}}
\newcommand{\LRa}{\ensuremath{\Leftrightarrow}}
\newcommand{\Ra}{\ensuremath{\Rightarrow}}
\newcommand{\La}{\ensuremath{\Leftarrow}}

%%% Pictures
\newlength{\pagepic}
\setlength{\pagepic}{390pt}

\newlength{\halfpagepic}
\setlength{\halfpagepic}{210pt}

\newcommand{\magicpicture}[1]{%
\ifpdf
  \input{pictures/#1.pdftex_t}
\else
  \input{pictures/#1.pstex_t}
\fi}

\newcommand{\diagram}[2]{%
\begin{figure*}
\begin{center}
\includegraphics[width=\pagepic]{pictures/#1}
\caption{{{#2}}}
\label{fig:#1}
\end{center}
\end{figure*}
}

\newcommand{\magicdiagram}[2]{%
\begin{figure*}
\begin{center}
\magicpicture{#1}
\caption{{{#2}}}
\label{fig:#1}
\end{center}
\end{figure*}
}

\newcommand{\twocaptiondiagram}[3]{%
\begin{figure*}
\begin{center}
\magicpicture{#1}
\caption[{{#2}}]{{{#3}}}
\label{fig:#1}
\end{center}
\end{figure*}
}

%%% Floating controls.  See discussion at
%%% http://dcwww.camd.dtu.dk/~schiotz/comp/LatexTips/LatexTips.html
\renewcommand{\topfraction}{0.85}
\renewcommand{\textfraction}{0.1}
\renewcommand{\floatpagefraction}{0.75}

\author{Alexey Radul}
\title{Cells, Compounds, and Closures}

\begin{document}

\maketitle

\begin{abstract}
How should one represent partial information over compound data?
This document describes the problem, the two alternatives, and why
the propagator system chose one of them.
\end{abstract}

\section{The Question}

Propagation of partial information is perfectly clear and sensible
when the partial information is over atomic base data.  A truth
maintenance system over numbers is a perfectly sensible object.  The
history of propagation and data flow systems has no major
disagreements about what to do with atoms.  Compound data, however,
present more of a challenge.  Let us consider what to do with partial
information over record structures (such as pairs made by \code{cons})
and over closures---these turn out to evoke interestingly different
intuitions about the right answer.  We will not consider compound
structures with intended invariants, such as unordered sets, because
there is enough trouble with just pairs and closures.

So, what's the problem?  Well, what should the \code{cons} propagator
do?  Clearly the output cell of the \code{cons} must learn that its
object is a pair, and that the information in the two input cells of
the \code{cons} is known about the respective fields of that pair.
Furthermore, subsequent manipulations of that resulting information
structure must be able to affect the pair as a whole: for example, the
output of
\begin{verbatim}
(e:switch (contingent #t '(bill)) (e:cons 1 2))
\end{verbatim}
is only known to be a pair if we believe \code{bill}.

There are two ways to arrange partial information over record
structures like \code{cons}: recursive partial information, and
partial information carrying cells.\footnote{Actually, there is also a
  third, that might be called flat partial information.  While
  conservative, it is grossly wasteful of information.  Since it is
  clearly a bad idea, and since the propagator thesis already
  discusses it, we will omit flat partial information from the present
  text.}  Recursive partial information is the strategy of copying the
partial information about the fields of the \code{cons} into those
fields, and emitting into the output a pair that recursively contains
whatever partial information was known about the fields.  For example,
the program
\begin{verbatim}
(define-cell x (e:cons (make-interval 1 2) (make-tms (contingent 'foo '(bill)))))
\end{verbatim}
would (eventually) put into the cell \code{x} a partial information
structure that is a pair, whose \code{car} field is an interval, and
whose \code{cdr} field is a one-entry TMS that contains a contingent
symbol.

Partial information carrying cells is the strategy of directly
grabbing the cells that contain the information about the fields of a
structure and emitting a structure whose fields carry those cells.
Under partial information carrying cells, the same program
\begin{verbatim}
(define-cell x (e:cons (make-interval 1 2) (make-tms (contingent 'foo '(bill)))))
\end{verbatim}
would (eventually) put into \code{x} a pair whose \code{car} was a
cell that contained an interval, and whose \code{cdr} was a cell that
contained a one-entry TMS.

Which to choose?

\section{The Options}

Recursive partial information is appealingly safe.  The contents of
every cell remains a data structure in the underlying implementation
system; merging such data structures can be done by recursively
calling \code{merge} because \code{merge} remains functional; the
effects of every propagator are perfectly local.  If such a data
structure acquires dependencies, for example by traveling through a
conditional, they can be attached at the outermost level, and any
attempt to access the gets will have to go through them.  Everything
looks clean and nice.

From the perspective of thinking about record structures (as opposed
to closures), there is just one downside: if a field of some record
structure gains new information, that new information has to be
propagated by local steps through every propagator that manipulates
the record structure, even though only the accessor at the end of this
chain is interested in what that information actually is.  And
unfortunately this effect is contagious: a field of a field of a field
must be pushed around through every record in the whole tower.  This
creates a large amount of uninteresting extra work.

Partial information carrying cells avoids the extra recopying work,
because an update into the cell holding information about a field of a
datastructure is immediately visible in any place where that cell has
been carried, without any effort on the part of the intermediate
propagators on the path.  Partial information carrying cells is also
intellectually appealing for the reason that propagator cells are
meant to be analagous to memory locations in a conventional
programming language, and the fields of a conventional record
structure are memory locations, so the fields of a propagator record
structure should, by symmetry, be cells.  Unfortunately, partial
information carrying cells also creates a host of semantic problems.
Those semantic problems will turn out to have an elegant solution, but
let us first enumerate them.

First and least troublesome, carrying cells around just feels a little
weird.  At the very least, we need to invent what a cell means when
interpreted as itself being a partial information structure.  But this
is a purely aesthetic concern.  Second, carrying cells introduces
creepy action at a distance: a cell can be carried arbitrarily far,
and then becomes a channel for instantaneous communication across that
distance.  This is somewhat strange, but on the other hand, this was
arguably the point, because the alternative is for that communication
channel to be an emergent property of lots and lots of boring
propagation of updates to fields of data structures.  There is an
interesting issue of how to make that channel properly respect
introduction of, say, dependencies on the way, but is subsumed by the
most pressing problem with partial information carrying cells.

The most pressing problem with partial information carrying cells is
the question of how to merge the resulting structures.  Certainly a
pair should be merged by merging its constituents; but how to merge
two propagator cells?

There is only one sensible way to merge two propagator cells.  If two
cells are being merged, that means the program has discovered that the
information they store is about the same entity.  Therefore, the thing
to do is to attach identity propagators to those two cells, to keep
their information equal in the future.\footnote{Theoretically, one
  could imagine detaching all the propagators that read and write one
  of them and attaching them to the other instead (and also copying
  all the information currently present in the loser into the winner).
  However, this is a pain to do when propagators are implemented as
  Scheme closures, and still suffers from all the problems described
  in the following text.}  The trouble is, this attachment of identity
propagators needs to be sensitive to the context in which the merge is
being done.  If, for example, the two cells in question are being
merged contingently on some premise, then the forwarding of
information from one to the other has to be kept contingent on the
same premise.  But how should this be arranged?  If \code{merge} is to
return the information structure that results from merging its two
arguments, then it must do any necessary attaching itself, as a side
effect.  But if it is to be called recursively from inside, say, a
TMS, then it must be pure, so that the TMS can attach the premises it
needs to attach to all the results of \code{merge}.  This conundrum,
which turns out to have a solution that proved elegant but took time
to think of, caused the propagator system described in the thesis to
prefer the recursive partial information strategy.

A note about closures.  A closure is a compound data structure whose
fields correspond to the free variables of the expression being
closed.  Therefore, the recursive partial information versus carrying
cells question applies to closures as well.  A cell-carrying closure
would contain the actual cells that were the free variables of the
expression; a recursive partial information closure would contain the
information present in those cells.  In the context of closures,
grabbing the cells themselves feels much nicer, because the expression
being closed may want to attach propagators to some of its free
variables (for example, to compute something about them from some of
its arguments); if so, the information those propagators compute
should flow to the original cells holding those variables, and thence
to any other closures or propagators that read those cells.  A closure
that carried partial information rather than cells would need to
always be treated bidirectionally by all the propagators that moved
the closure itself around to arrange this, whereas a closure that
carries cells provides that bidirectional channel automatically.
\todo{Discuss the pros and cons of having cell merge produce
  bidirectional links.  Leads to automatic unification of stuff
  carried around in data structures --- if this ever undesirable?  Can
  it be shut off?  (Maybe by manually inserting a unidirectional
  copying frob somewhere...)}  Also, the \code{apply} propagator would
have to do something interesting with recursive information closures,
because if new partial information became available about the closed
variables of a closure that had already been applied, it would need to
be communicated into (and out of) the network created by that
application.  We never did figure out a good way to arrange this, so
the system built around recursive partial information did not have
closures.

\section{The Answer}

The main problem with recursive partial information is essential to
it: If \code{cons} has a purely local effect on information
structures, then updating a field will require updating the compound.
The main problem with partial information carrying cells, however, is
conceptual: what does a cell mean as partial information?  How to
merge cells in a functional way?  It turns out that the conceptual
problem has a solution.

What is a propagator cell, really?  It is a locus for storing all
information known about some program entity.\footnote{Where a memory
  location in a traditional programming language is a way to hold on
  to a (fixed, known) program entity.}  Adding information to a cell
is saying that the information being added is known to describe the
entity that cell is responsible for.  If the cell already had some
information about that entity, it concludes that \emph{both} of those
pieces of information are known about that entity, and it must merge
them to compute the new best information known about that entity.

The story just presented suggests the following contract between
\code{merge} and the cell: \code{merge} answers the question
\begin{quote}
``What is the least-commitment information structure that captures
all the knowledge in these two information structures?''
\end{quote}
and the cell updates itself to contain the answer.  This \code{merge}
is a pure function, and is therefore nice to deal with, including
calling it recursively to handle partial information structures that
contain other partial information structures.  However, this question
has no answer when the structures being merged are themselves cells.
A cell as partial information is not really just a pile of knowledge;
it is something else.  Merging two cells does not cleanly produce a
new pile of knowledge, because it must somehow connect the cells.
This story is not adequate.

Let us return to the propagator cell, then.  It is a locus for storing
all information about some program entity, and adding information to
it constitutes asserting that the new information also describes that
entity.  But merely functionally merging information, and updating the
cell to hold the merged information, is not enough to describe all the
possible consequences of that discovery.  Learning that two
cells-as-information are actually about the same object, for example,
necessitates synchronizing them.

The solution is to refine the contract between \code{merge} and the
cell.  Now \code{merge} answers the question
\begin{quote}
``What needs to be done to the network in order to make it reflect the
discovery that these two information structures are about the same
object?''
\end{quote}
and the cell executes the answer.  This subsumes the previous
interface, because the answer can just be ``update the cell
responsible for that object to contain the following information
structure.''  But now there is room for other answers: ``synchronize
these two cells,'' or ``signal these premises as a nogood set.''

With that question in mind as the question \code{merge} answers,
\code{merge} can still be a pure function that can be used
recursively.  The only extension is that it may return a data
structure representing a side-effect that may need to be performed on
the network, and any partial information structure whose merging
recursively calls \code{merge} must take care to do the right things
with those representations of side-effects.  But of course, with
responsibility comes power: doing it this way gives that partial
information structure the opportunity to adjust the side-effect to
reflect the context in which it was called for.

For example, suppose we are merging one pair contingent on the premise
\code{bill} with another contingent on \code{fred}.  The code for
merging the contingent objects intends to recursively merge the pairs
and then attach the \code{bill} and \code{fred} dependencies to them.
Merging the pairs is the assertion that the two pairs represent the
same object, so their corresponding fields also represent the same
objects; therefore we must recursively merge them.  Each field in each
pair is a cell; each of these merges produces an object that requests
that the corresponding cells be synchronized by connecting them with
identity propagators.  The merge of the pairs concatenates those
effects into one side-effect object that represents a request to do
both of those things.  But now that side-effect-request object is
handed to the contingency merge.  The thing the contingency merge must
produce has to capture the idea that these two pairs we started with
may represent the same object, but we are only sure of that if the
\code{bill} and \code{fred} premises are believed.  In particular, all
the consequences of merging those cells in that pair must be made
contingent on those two premises.  One way to do that is, instead of
attaching identity propagators to the cells to synchronize them, we
can attach bidirectional \code{switch} propagators to synchornize the
contents only if both the desired premises are believed.  Because of
the way \code{switch} operates, the premises will get attached to the
transmitted data, and the deductions made from the transmission along
this bridge will remain contingent on those premises.

Adopting this interface also solves a long-standing bug we had with
our TMSes.  We had built our TMSes to signal nogood sets with a
side-effect that occurred when the TMSes were merged.  This worked
fine as long as the TMSes were always over atomic data, but would give
the wrong answers if we had a pair of TMSes over compound structures
that contained TMSes over fields, because any nogoods signalled by the
recursive merges of the interior TMSes would be missing the
dependencies carried for the structures themselves by the exterior
ones.  But now, signalling a nogood set can also be a side-effect
requested by a merge, which is returned until is gets all the way to
the cell that actually initiated the merging; and so the exterior TMS
has a chance to intercept it and attach more premises to it.

\section{The Consequences}

The meaning of a propagator cell as a partial information structure
now becomes clear.  A cell is a locus for strong all information
available about some program entity.  When one cell contains another,
this is an indirection: it says ``The information about this entity is
actually stored in \emph{this} cell.''  When two such cells are
written as partial information into the same container cell, the
meaning is ``\emph{Both} of these cells store information about my
entity,'' and those cells must be connected.  This would happen most
naturally if the entity one cell was responsible for was actually a
pair, and two different cells were found to be responsible for its
\code{car}, say.

With this way of thinking about cells as partial information, we can
further subdivide \code{cons}.  Instead of worrying about the strategy
of \code{cons}, we can define a unary primitive called, say,
\code{deposit}, that picks up its single argument cell and writes it
as partial information into the cell holding its return value.  Then
the partial information carrying cells strategy can be seen as a use
pattern of the recursive partial information strategy, where all the
fields of every \code{cons} propagator are required to be
\code{deposit}ed cells, rather than any other partial information.
Pleasantly, the inverse of \code{deposit} is also \code{deposit},
because of the bidirectionality of the connections created by merging
cells.  Accessors like \code{car}, however, do need to know whether
the relevant field of the record being accessed is a cell, because if
it is an inverse \code{deposit} suffices, but if it is not, they must
actually access that field and write the result to their output cell.

With the problem of merging cells solved, the only major obstacle to
implementing closures disappears.  Just grab the cells holding the
free variables and go.  Merge closures by checking that their
expressions are equivalent (read: identical) and by merging the cells
they close over.

With this new contract for \code{merge} in place, the cell gains a
clear role as the locus of side effects.  Only a cell may initiate a
side-effect on the structure of the network, whereas the \code{merge}
operation may not; but any cell may do so.  There is no single
\code{IO} monad to thread through everything here; no hot-spot
filtering all side-effects and imposing a total order on them.  Side
effects on separate portions of the network structure can and should
interleave in unspecified order, to avoid imposing spurious
dependencies by specifying it.  It is up to the monotonicity of
information to ensure that these ``race conditions'' at the network
structure level do not cause non-deterministic behavior at the
information level.

Record structures that carry propagator cells that contain more
records that carry more cells are like large record piles in
conventional languages, but with an extra level of indirection.  The
emergent behavior of merging such partial record structures is
unification, with the carried cells serving the role of the
unification variables.  Merging obeys the indirection, so merging two
different cells differs from unifying two variables: instead of
recurring on the structures they contain immediately, merging cells
just attaches synchornization propagators to them.  As the network
continues to run, the structures those cells contain will be merged,
but a single \code{merge} will not do an unbounded amount of work.
The indirection also removes the need for an occurs check:
violating the occurs rule produces not an infinite but a cyclic data
structure.  Representing the cyclic structure causes no trouble in
itself, and leaves the reader with the opportunity to interpret the
cycle usefully.

\end{document}
