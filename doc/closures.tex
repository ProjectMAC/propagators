\documentclass[12pt,letterpaper]{article}

\input{preamble.tex}
\author{Alexey Radul}
\title{Cells, Compounds, and Closures}

\begin{document}

\maketitle

\begin{abstract}
How should one represent partial information over compound data?
This document describes the problem, the two alternatives, and why
the propagator system chose one of them.
\end{abstract}

\section{What the problem is}

Propagation of partial information is perfectly clear and sensible
when the partial information is over atomic base data.  A truth
maintenance system over numbers is a perfectly sensible object.  The
history of propagation and data flow systems has no major
disagreements about what to do with atoms.  Compound data, however,
present more of a challenge.  Let us consider what to do with partial
information over record structures (such as pairs made by \code{cons})
and over closures---these turn out to evoke interestingly different
intuitions about the right answer.  We will not consider compound
structures with intended invariants, such as unordered sets, because
there is enough trouble with just pairs and closures.

So, what's the problem?  Well, what should the \code{cons} propagator
do?  Clearly the output cell of the \code{cons} must learn that its
object is a pair, and that the information in the two input cells of
the \code{cons} is known about the respective fields of that pair.
Furthermore, subsequent manipulations of that resulting information
structure must be able to affect the pair as a whole: for example, the
output of
\begin{verbatim}
(e:switch (contingent #t '(bill)) (e:cons 1 2))
\end{verbatim}
is only known to be a pair if we believe \code{bill}.

There are two ways to arrange partial information over record
structures like \code{cons}: recursive partial information, and
partial information carrying cells.\footnote{Actually, there is also a
  third, that might be called flat partial information.  While
  conservative, it is grossly wasteful of information.  Since it is
  clearly a bad idea, and since the propagator thesis already
  discusses it, we will omit flat partial information from the present
  text.}  Recursive partial information is the strategy of copying the
partial information about the fields of the \code{cons} into those
fields, and emitting into the output a pair that recursively contains
whatever partial information was known about the fields.  For example,
the program
\begin{verbatim}
(define-cell x (e:cons (make-interval 1 2) (make-tms (contingent 'foo '(bill)))))
\end{verbatim}
would (eventually) put into the cell \code{x} a partial information
structure that is a pair, whose \code{car} field is an interval, and
whose \code{cdr} field is a one-entry TMS that contains a contingent
symbol.

Partial information carrying cells is the strategy of directly
grabbing the cells that contain the information about the fields of a
structure and emitting a structure whose fields carry those cells.
Under partial information carrying cells, the same program
\begin{verbatim}
(define-cell x (e:cons (make-interval 1 2) (make-tms (contingent 'foo '(bill)))))
\end{verbatim}
would (eventually) put into \code{x} a pair whose \code{car} was a
cell that contained an interval, and whose \code{cdr} was a cell that
contained a one-entry TMS.

Which to choose?

\end{document}
