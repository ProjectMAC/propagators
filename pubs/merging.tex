\documentclass[12pt]{article}

\usepackage{lisp-env}
%%%% Alexey's general-purpose LaTeX stuff

%% Paper size is a persistent annoyance.  The (current) suggested
%% strategy is to always use letterpaper or a4paper options to
%% documentclass, and see to it that \pdfpagewidth and \pdfpageheight
%% get set correctly.  The color package, for example, does the
%% latter.

%% Also be sure to view the document at the right paper size with
%% xdvi, and print it at the right paper size with dvips.

\usepackage{amsmath}
\usepackage{amsthm}
\usepackage{amsfonts}
\usepackage{amssymb}

\usepackage{epsfig}
\usepackage{graphicx}
\usepackage[usenames]{color}

\usepackage{ifpdf}

\newtheorem{theorem}{Theorem}[section]
\newtheorem{lemma}[theorem]{Lemma}
\newtheorem{corollary}[theorem]{Corollary}
\newtheorem{statement}[theorem]{Statement}
\newtheorem{conjecture}[theorem]{Conjecture}

\ifpdf
  \DeclareGraphicsRule{.pdftex}{pdf}{.pdftex}{}
\else
\fi

%%% Commands
\newcommand{\code}[1]{{\tt #1}}
\newcommand{\defn}[1]{{\bfseries{#1}}}
\newcommand{\todo}[1]{{\bf *** TODO: {#1} ***}}
%% The idea here is to mark todo items that will not be showstoppers
%% if the todo comment just vanishes.
\newcommand{\otodo}[1]{{\bf *** TODO (Optional): {#1} ***}}
% \renewcommand{\otodo}[1]{{}}  % Like so.

\def\caret{\^\,\hspace{-2pt}}

%%% Shortcuts
\newcommand{\be}{\begin{equation}}  %numbered equations
\newcommand{\ee}{\end{equation}}
\newcommand{\bea}{\begin{eqnarray}}
\newcommand{\eea}{\end{eqnarray}}
\newcommand{\beas}{\begin{eqnarray*}}
\newcommand{\eeas}{\end{eqnarray*}}
\newcommand{\bl}{\begin{lisp}}
\newcommand{\el}{\end{lisp}}
\newcommand{\bls}{\begin{lisp*}}
\newcommand{\els}{\end{lisp*}}
\newcommand{\bv}{\begin{verbatim}{}}
\newcommand{\ev}{\end{verbatim}{}}
\newcommand{\LRa}{\ensuremath{\Leftrightarrow}}
\newcommand{\Ra}{\ensuremath{\Rightarrow}}
\newcommand{\La}{\ensuremath{\Leftarrow}}

%%% Pictures
\newlength{\pagepic}
\setlength{\pagepic}{390pt}

\newlength{\halfpagepic}
\setlength{\halfpagepic}{210pt}

\newcommand{\magicpicture}[1]{%
\ifpdf
  \input{pictures/#1.pdftex_t}
\else
  \input{pictures/#1.pstex_t}
\fi}

\newcommand{\diagram}[2]{%
\begin{figure*}
\begin{center}
\includegraphics[width=\pagepic]{pictures/#1}
\caption{{{#2}}}
\label{fig:#1}
\end{center}
\end{figure*}
}

\newcommand{\magicdiagram}[2]{%
\begin{figure*}
\begin{center}
\magicpicture{#1}
\caption{{{#2}}}
\label{fig:#1}
\end{center}
\end{figure*}
}

\newcommand{\twocaptiondiagram}[3]{%
\begin{figure*}
\begin{center}
\magicpicture{#1}
\caption[{{#2}}]{{{#3}}}
\label{fig:#1}
\end{center}
\end{figure*}
}

%%% Floating controls.  See discussion at
%%% http://dcwww.camd.dtu.dk/~schiotz/comp/LatexTips/LatexTips.html
\renewcommand{\topfraction}{0.85}
\renewcommand{\textfraction}{0.1}
\renewcommand{\floatpagefraction}{0.75}

\usepackage{url}

\title{Cumulative Programming}
\author{Alexey Radul}

\begin{document}

\maketitle

\begin{abstract}
We introduce \emph{cumulative programming}---a programming style
founded on the idea that it can often be fruitful to replace the
notion of a ``value'' with the notion of a repository of all available
information about that value.  Each such repository accepts
information from a variety of sources, and is equipped with a merging
operation that combines the information already in the repository with
the new information tendered.  This allows incremental refinement of
the value in question as more information becomes available; several
such repositories can interact to mutually refine as a natural part
of the operation of a program.

Cumulative programming with sufficiently well-behaved merge operations
lends itself to concurrent semantics and execution.  The accumulation
repositories are the locus of synchronisation (locking), and are
fundamentally local, so they do not create serious contention.
Furthermore, if the merge operations are commutative, then writer-writer
race conditions become unimportant, and if they are monotonic and
idempotent, then reader-writer race conditions also become
unimportant.

We illustrate the cumulative programming style with two examples:
inferences a (toy!) compiler might perform about values in a program being
compiled, and enforcement of arc consistency in a constraint network a
la Waltz~\cite{?}.
\end{abstract}

\section{Introduction}

Sometimes a program needs to accumulate information about something
over the course of several perhaps complex operations.  A reporting
application needs to collate the results of multiple database queries
to produce an answer.  An optimizing compiler needs to learn how
variables may be used, or where functions may be called.  Sometimes
new knowledge gleaned about one thing helps learn about another, and
sometimes two things may mutually inform each other; perhaps a long
back-and-forth is needed until they converge.  In any compiler, data
flow analysis learns from control flow analysis; and in a compiler for
a higher-order language, control flow analysis learns from data flow
analysis (where a data object representing a procedure might flow
affects whence control might flow into that procedure)
\cite{shivers-1991-taming-lambda}.

Sometimes, different methods need to be tried, and it may not be
possible to predict which one, or what combination, will work, until
they are tried, and until they report either new discoveries or lack
thereof.  A symbolic integrator can try many particular integration
patterns, but each one will only work on some kinds of inputs.  An
input range may be given by both endpoints, or by one end and a
length.  Sometimes, it may be well to try different methods
concurrently, and perhaps stop one if it enters a long (or infinite)
computation that another method makes unnecessary.  A finite list is easy
to search, but otherwise-expensive breadcrumbs to remember where one
has been may save one from running down a circular list forever.

These patterns are so pervasive that perhaps we don't think about
them; and yet there is a common thread that bears further examination,
and explicit acknowledgement and support.  All these cases, whether as
complicated as lofty interwoven compiler analyses or as simple as the
lowly range bar that accepts two different forms of input look, from
the right angle, like systems of accumulators.  It doesn't matter in
what order a compiler analysis discovers what it does about the uses
of some procedure, as long as it can maintain an accurate summary of
the whole.  It doesn't matter if two different integration methods
yield results in different forms, as long as they can be reconciled
and the best form chosen for subsequent efforts.

There are three properties we mentally associate with the idea of an
accumulator, all of which are independently useful.  One is
\emph{commutativity}: it doesn't matter in what order one add information to
an accumulator, because the end result is the same.  This is very
useful because we then don't need to choreograph the order in which
information is discovered: we can try whatever methods look plausible
willy-nilly (perhaps even concurrently), and have the accumulation
take care of making sure the answer available at the end is coherent.
Another is \emph{monotonicity}: an accumulator, by definition, accumulates
everything added to it, and never forgets anything that might be
valuable.  This means we don't need to worry about reading the state
of some accumulator before it is overwritten, because of course it
will not be overwritten---only added to.  Anything there is to be had
by looking at it will still be there later, so we can allow other
writers to proceed before readers do.  Finally the third common
property is \emph{idempotence}: if one tells an accumulator the same piece of
information twice, it should not double-count.  This property is
valuable because it allows us not to worry about reading our
accumulators too early: If it later learns something new, we can just
read it again, and repeating our conclusions to the next accumulator
down the line won't cause any trouble, because it will be able to
recognize and ignore any redundnacy such a repetition may introduce.
These three ideas together form the essence of cumulative programming.
As with object-oriented programming, these ideas can be built into the
core of a programming system, as in \cite{art-thesis}, or used as a
% TODO I want to suggest that the former is a good thing, not a crazy
% extremist goose chase.
design pattern to simplify tasks in existing environments.

\section{Example: Shortest Paths}

One simple example of cumulative programming is finding
(single-source) shortest paths in graphs.  As a reminder, the problem
is this: given a (directed) graph $G$ with weighted edges, and a
distingushed source vertex $s$, find the shortest (least total weight)
paths from $s$ to all other vertices (or to a distinguished sink
vertex $t$).  The traditional form in which the answer is expected is
an association that maps each vertex to the length of the shortest
path from the source to that vertex, and the vertex's predecessor in
that path.

The textbook exposition of how to solve this problem is an exercise in
cumulative programming.  For every vertex of the input graph, let
there be an accumulator tracking two things: the length of the
shortest known path from the source to this vertex, and this vertex's
predecessor along that path.  Each (directed) edge of the input graph
allows us to deduce a proposition for the head vertex from the state
of the accumulator for the tail.  Each accumulator knows whether to
accept or reject such propositions, based on whether the proposed new
length is more or less than the best length that accumulator has seen
so far.  Therefore, to solve the shortest paths problem, it suffices
to walk over said edges deducing said propositions until no
improvements remain to be deduced.

Both of the standard shortest-paths algorithms refine this base with a
prescription for how to walk the edges of the graph.  The Bellman-Ford
algorithm says to go through all edges $|V|-1$ times (where $|V|$ is
the number of vertices of the graph), and promises that if the graph
has no negative-weight cycles, $|V|-1$ times around is enough to
deduce all possible improvements.  Dijkstra's algorithm requires the
absence of negative-weight edges, and promises that if each time one
processes all the edges leaving the closest-to-the-source vertex whose
edges have not yet been processed, one will only need to process each
vertex and each edge once.

Let us examine this situation from the cumulative programming
perspective.  We notice a separation between the maintenance of the
information we are trying to compute and the schedule of our
computations.  Our accumulators are completely self-contained
entities: each receives assertions of the form ``if vertex X were your
predecessor, you could have a path as short as Y,'' from any source,
and remembers from these assertions the best offered length, and the
associated predecessor.  The order in which we compute these
assertions is independent of their treatment by the
accumulators---there is a separation of concerns.

How does this structure rate on the cumulative programming triumvirate
of commutativity, idempotence, and monotonicity?  The accumulation is
commutative: if a shorter and a longer path are presented, the shorter
will be chosen, regardless of which was presented first.  Therefore,
we are free to compute and present possible paths to a particular
vertex in any order we choose---the accumulator will sort out the best
one.  The accumulation is monotonic, in the sense that the quality of
the best path always improves (or stays the same, but never worsens):
if a longer path is offered when a shorter is already present, the
longer will just be rejected.  Therefore, we are free to wait, as
Dijkstra's algorithm does, to collect all the data about short paths
to a vertex before we start computing paths from that vertex---the
intermediate possibilities are of no concern for the problem at hand.
The accumulation is idempotent: if a path that was presented before is
presented again, it will have no effect, because the best path will
necessarily be no worse.  Therefore, we are free, per the Bellman-Ford
algorithm, not to wait to be sure we know the shortest path to a
vertex before computing paths through it, because the worst that can
happen is that we will need to redo work we have already done if we
discover some new, shorter path; and on the other hand, having to wait
for that would lead to paralysis in the presence of negative-weight
edges.

The Bellman-Ford shortest paths algorithm particularly makes an
excellent showcase of the general advantages of cumulative
programming.  When it starts, we don't know the exact shortest path to
any vertex except the source; and we don't have enough information to
deduce any exact shortest paths to any vertex, because negative-weight
edges elsewhere in the graph can always produce a surprisingly short
path.  There is no way to construct any more of the final answer
irreversibly.  So we start making progress with tentative deductions:
there is a path to vertex $v$ of length foo (so the shortest cannot be
any longer than that); therefore, there is a path to vertex $w$ of
length bar; etc.  Each of these things is true and correct, though it
remains to be seen whether it will prove to be the last word.  If we
find a shorter path to an old vertex, we will have to redo the
deductions we made about paths through that vertex in the past, but
that's OK.  We are continually learning; accumulating information
about the problem we are trying to solve.  When there is no more
information to be accumulated, the problem is solved; and in the case
of shortest paths through graphs, it is even known that if paths keep
getting shorter for too long, the graph has a negative-weight cycle
and the shortest-paths problem has no solution.

% Waltz
% Compiler analysis

\bibliographystyle{plain}
\bibliography{merging-auto}

\end{document}

N.B. I can even write all the examples in three different languages
for three different audiences.

If merge is commutative, you don't have writer-writer race conditions,
If it's monotonic, you don't need to worry about reading too late
It it's idempotent, you don't need to worry about reading too early,
  because you can always reread and recompute

Do I want the ``people think multidirectionally'' story?

Just how much does this story help you avoid locks?  Certainly a
writer must lock the cell being written to, at least for a moment; but
are there any other problems of this sort?  How much do monotonicity
and idempotence help you not need transactions?

Examples:

Know/don't know (maybe ``I've tried X Y and Z and I still don't know yet'')
bounds (the improving values paper, Dijkstra's algorithm, etc)
numeric ranges for dealing with measurements
pointwise bounds on vectors?
dependency tracking as an information-type transform
Waltz-style restriction of possibilities
Type information (or flows-to lists, etc) for program analysis
Logical statements
Gerry says: build-system dependency lists


Implementational notes:

N.B. The partial information structure itself wants to be immutable, so
that all changes to it travel through the cell interface.
(n.b. Immutability allows structure sharing.)  But it may want to
contain other cells (and maybe even propagate change notices).  (Does
immutability and structure sharing require garbage collection?)

N.B. ``More informative'' has no reason to mean ``takes more memory''
(e.g. [5,8] is more informative than [2,10])

N.B. If you're going to have observers, the observers should only be
notified when *stored information changes*, not when *new increments
arrive* (because they may be redundant).


The lowly counter:

Read-Modify-Write loses.
If incrementation can be captured as an operation, then said incrementation
commutes with itself: this is commutativity of merging
Next level is to replace the counter by a set whose size the counter
is, and add elements with identities to it.  This gives commutativity
and idempotence, albeit at an exponential cost in memory.  That memory
cost can be mitigated by garbage collection: have the set hold its
elements weakly, and add a GC hook that converts dead weak links into
a count of how many of them there were.  In effect, this is the same
counter again, but it now maintains a weak cache of the identities
that have been added to it.  One nice thing about idempotence is that
you can send an idempotent message out of a transaction immediately,
without having to wait for the transaction to commit, and just resend
the message if you need to retry the transaction (but you lose if the
transaction fails).
