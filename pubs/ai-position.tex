\documentclass[12pt]{article}

% \usepackage{lisp-env}
%%%% Alexey's general-purpose LaTeX stuff

%% Paper size is a persistent annoyance.  The (current) suggested
%% strategy is to always use letterpaper or a4paper options to
%% documentclass, and see to it that \pdfpagewidth and \pdfpageheight
%% get set correctly.  The color package, for example, does the
%% latter.

%% Also be sure to view the document at the right paper size with
%% xdvi, and print it at the right paper size with dvips.

\usepackage{amsmath}
\usepackage{amsthm}
\usepackage{amsfonts}
\usepackage{amssymb}

\usepackage{epsfig}
\usepackage{graphicx}
\usepackage[usenames]{color}

\usepackage{ifpdf}

\newtheorem{theorem}{Theorem}[section]
\newtheorem{lemma}[theorem]{Lemma}
\newtheorem{corollary}[theorem]{Corollary}
\newtheorem{statement}[theorem]{Statement}
\newtheorem{conjecture}[theorem]{Conjecture}

\ifpdf
  \DeclareGraphicsRule{.pdftex}{pdf}{.pdftex}{}
\else
\fi

%%% Commands
\newcommand{\code}[1]{{\tt #1}}
\newcommand{\defn}[1]{{\bfseries{#1}}}
\newcommand{\todo}[1]{{\bf *** TODO: {#1} ***}}
%% The idea here is to mark todo items that will not be showstoppers
%% if the todo comment just vanishes.
\newcommand{\otodo}[1]{{\bf *** TODO (Optional): {#1} ***}}
% \renewcommand{\otodo}[1]{{}}  % Like so.

\def\caret{\^\,\hspace{-2pt}}

%%% Shortcuts
\newcommand{\be}{\begin{equation}}  %numbered equations
\newcommand{\ee}{\end{equation}}
\newcommand{\bea}{\begin{eqnarray}}
\newcommand{\eea}{\end{eqnarray}}
\newcommand{\beas}{\begin{eqnarray*}}
\newcommand{\eeas}{\end{eqnarray*}}
\newcommand{\bl}{\begin{lisp}}
\newcommand{\el}{\end{lisp}}
\newcommand{\bls}{\begin{lisp*}}
\newcommand{\els}{\end{lisp*}}
\newcommand{\bv}{\begin{verbatim}{}}
\newcommand{\ev}{\end{verbatim}{}}
\newcommand{\LRa}{\ensuremath{\Leftrightarrow}}
\newcommand{\Ra}{\ensuremath{\Rightarrow}}
\newcommand{\La}{\ensuremath{\Leftarrow}}

%%% Pictures
\newlength{\pagepic}
\setlength{\pagepic}{390pt}

\newlength{\halfpagepic}
\setlength{\halfpagepic}{210pt}

\newcommand{\magicpicture}[1]{%
\ifpdf
  \input{pictures/#1.pdftex_t}
\else
  \input{pictures/#1.pstex_t}
\fi}

\newcommand{\diagram}[2]{%
\begin{figure*}
\begin{center}
\includegraphics[width=\pagepic]{pictures/#1}
\caption{{{#2}}}
\label{fig:#1}
\end{center}
\end{figure*}
}

\newcommand{\magicdiagram}[2]{%
\begin{figure*}
\begin{center}
\magicpicture{#1}
\caption{{{#2}}}
\label{fig:#1}
\end{center}
\end{figure*}
}

\newcommand{\twocaptiondiagram}[3]{%
\begin{figure*}
\begin{center}
\magicpicture{#1}
\caption[{{#2}}]{{{#3}}}
\label{fig:#1}
\end{center}
\end{figure*}
}

%%% Floating controls.  See discussion at
%%% http://dcwww.camd.dtu.dk/~schiotz/comp/LatexTips/LatexTips.html
\renewcommand{\topfraction}{0.85}
\renewcommand{\textfraction}{0.1}
\renewcommand{\floatpagefraction}{0.75}

\usepackage{url}

\title{On Propagation Networks as a Substrate for Artificial Intelligence Programming}
\author{Alexey Radul}

\begin{document}

\maketitle

\section{Introduction}

Modern programming languages are woefully inadequate for the needs of
the artificial intelligence field.

% AI and PL both aim at the language of thought
Both the fields of artificial intelligence and programming languages
aim for the language of thought.  Artificial intelligence is always on
a quest for better representations of knowledge---what better
representation for procedural knowledge than a program?  But in what
language should that program be expressed?  That language will be the
language of thought for the system that represents its knowledge in
it!

Programming language research is always on a quest for precise
languages with which to describe processes and computations.  One of
the great goals of this quest is to find languages that better match
how people think about the world, so that programs will be more
natural, more understandable, and therefore easier to think about and
debug.  The holy grail is again the same: to find the language in
which we think.

% rant about backchannels not being thick enough

% multidirectionality of thought
Multidirectionality is a particular feature of thought that is visibly
lacking from modern programming languages.  When we think about
something, we can pull information and deductions about it from many
mental sources.  A chair is an object, so it will fall if I drop it.
A chair is a built artifact with a purpose, and its purpose is to
support people sitting on it.  Two completely different bits of
knowledge, from reasoning processes in two different domains, and yet
they apply to the same chair.  And they interact: if I drop the chair
from high enough, it may break, and become unservicable for its
intended purpose---people who sit on it thereafter may themselves
fall.  It also doesn't matter much in what order those two
observations came in.  In either case, I have the same resulting set
of ideas about the chair.

Standard programming languages, in contrast, are singularly
unidirectional.  The time in a computation flows inexhorably in one
direction.  The successful
evaluation of an individual expression, or, equivalently, the successful
production of an individual value, inescapably marks a point in time.  The work
done to produce that value came \emph{before}; the work that will be
done using that value comes \emph{after}.
That inevitable limitation manifests as the sequencing of operations
by the evaluator.  Since the evaluator works by structural recursion
on the expressions that comprise a program, the structure of the
program text constrains the flow of time through that program.
Thus, while a program must serve as the description of
\emph{what} we want accomplished, its shape excessively constrains the
process, and therefore \emph{how} the computer accomplishes it.


\section{Propagation Networks}

% multidirectionality; merging

\section{Applicability of Propagation to Artificial Intelligence}

% multidirectionality
% filling in details
% compatible with probabilistic approaches
% thought vs action
% propagators as critics and selectors
% multidirectionality and concurrency enables self-observation and self-awareness

% the cells are topic-specific blackboards

\end{document}
