\documentclass[12pt]{article}
\pagestyle{plain}

\usepackage{epsfig}
\usepackage{amsfonts}
\usepackage{amssymb}
\usepackage[usenames]{color}

\title{Functional Reactivity by Propagation}
\author{Alexey Radul}

\newcommand{\code}[1]{{\tt #1}}
\newcommand{\todo}[1]{{\bf *** TODO: {#1} ***}}

\begin{document}

\maketitle

\begin{abstract}
Functional Reactive Programming \cite{elliott-hudak-1997-fran, wan-hudak-2000-frp-first-principles,
  nilsson-2002-frp, cooper-2008-thesis} is a programming style characterized
by writing time-varying output signals (such as animations on a
display) as pure functions of time-varying input signals (such as the
position of a mouse), and relying on a runtime system to automatically
update the outputs in response to changes in the inputs.  As Cooper
observes in \cite{cooper-krishnamurthi-2006-frtime, cooper-2008-thesis}, the update portion of
a functional reactive system can naturally be viewed as a process of
propagating the consequences of changes in particular inputs; that is,
only updating those intermediate and final quantities that actually
depend on the inputs that changed (perhaps transitively, through a
chain of such dependencies).  The functional reactive system FrTime
\cite{cooper-krishnamurthi-2004-frtime,
  cooper-krishnamurthi-2006-frtime, cooper-2008-thesis} therefore contains
a special-purpose propagation subsystem tailored specifically to its
needs.

Radul and Sussman describe a general-purpose propagation
infrastructure in \cite{art}.  We exercise that
infrastructure here by implementing a functional reactive updating
subsystem on top of it.  The approach does not mimic FrTime's exactly,
and we discover a different set of tradeoffs.  By adhering to the
expectations of the general-purpose propagator \cite{art}, we
reap its benefits; in particular, this updater naturally supports
multiple semantically overlapping read-write views of the same
reactive objects --- a common challenge \cite{?} for functional
reactive systems.
\end{abstract}

\bibliographystyle{plain}
\bibliography{paper}

\end{document}

Introduce: propagation is general purpose; frp is hot these days;
let's implement it in it.

Summarize the propagation presented in Art

Summarize FRP

Describe how (the appropriate portion of) FRP fits into propagation

Show off advantages of multidirectionality

Discuss premise granularity trade-off

Dedicated related work section?

Conclusion.
